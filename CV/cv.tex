%!TEX TS-program = xelatex
\documentclass[]{friggeri-cv}
\usepackage{xeCJK}
\CJKsetecglue{}   % 去掉中英文之间的空格
%\newfontfamily{\kaiti}{Adobe Kaiti Std}
%\newfontfamily{\song}{STSong}
%\newfontfamily{\hei}{Adobe Heiti Std}
%\newfontfamily{\fsong}{Adobe Fangsong Std}
%\newfontfamily{\kaibd}{Kaiti SC Bold}
\defaultfontfeatures{Mapping=tex-text}
\hyphenation{op-tical net-works semi-conduc-tor}


\addbibresource{bibliography.bib}

\begin{document}
\header{}{彭超}
       {资深计算机视觉研究员}

% In the aside, each new line forces a line break
% 电话: 13121627145
\begin{aside}
  \section{联系方式}
    mikejay0520@163.com
    \href{http://www.pengchao.org}{www.pengchao.org}
  \section{研究方向}
    图像分类
    物体检测
    语义分割
    视频分析
\end{aside}

\section{教育经历}
\begin{entrylist}
  \entry
    {2014-2017}
    {硕士~~~~清华大学}
    {北京}
    {软件工程大数据方向}
  \entry
    {2010-2014}
    {学士~~~~武汉大学}
    {湖北}
    {空间信息与数字技术专业}
\end{entrylist}


\section{工作经历}
\begin{entrylist}
  \entry
    {2018.12 - 至今}
    {Momenta, 北京}
    {Senior Computer Vision Engineer}
    {负责无人驾驶视觉感知工作}

  \entry
    {2016.4 - 2018.11 \qquad }
    {Megvii (Face++), 北京}
    {Researcher}
    {在Megvii实习和工作期间,我主要负责通用物体分割、通用物体检测和图像分类等三个领域的研究和应用落地工作,取得了举世瞩目的成绩。}

\end{entrylist}

\section{获奖经历}
  \begin{entrylist}
  \entry
    {2018.09}
    {COCO \& Mapillary 2018 Challenge}
    {\href{http://cocodataset.org/workshop/coco-mapillary-eccv-2018.html}{coco-mapillary-eccv-2018}}
    {在本次比赛中,我们团队获得了四项冠军,分别是COCO Instance Segmentation, COCO Keypoint, COCO Panoptic Segmentation和Mapillary Panoptic Segmentation。我作为两个Panoptic Segmentation Task的leader和主要贡献者,我带领了我们组的五个实习生,统筹和设计了Panoptic Segmentation的算法流程。在比赛过程中,我实现了一套非常巧妙的融合策略,基本解决了object互相遮挡的问题,超过第二名5个点,同时也超越了Human Consistency,
    在COCO数据及上做到了极致。}

  \entry
    {2017.10}
    {COCO 2017 Challenge}
    {\href{https://places-coco2017.github.io/}{places-coco2017.github.io}}
    {在本次比赛中,我们团队获得了通用物体检测和人体骨骼关键点检测两项比赛冠军,通用物体分割亚军。我作为通用物体检测比赛的三名主要贡献者之一,研发了一套基于16台GPU机器集群的MegDet算法,一举夺得了冠军,超过亚军1.7个点,远胜Microsoft、Google、Facebook等公司的参赛队伍。相关工作已经总结投稿于CVPR 2018上。}

  \entry
    {2016.10}
    {PASCAL VOC Semantic Segmentation}
    {\href{http://host.robots.ox.ac.uk:8080/leaderboard/displaylb.php?challengeid=11&compid=6}{pascal\_voc\_2012}}
    {PASCAL VOC 2012是图像语义分割研究中常用的公开数据集,我在Megvii实习期间主要研究如何提高语义分割的极限性能,并于2016.10刷到榜单第一,相关工作已经总结发表在CVPR 2017上。}
  \end{entrylist}

\section{会议论文}
\href{https://arxiv.org/abs/1903.05027}{An End-to-End Network for Panoptic Segmentation}\newline
Huanyu Liu, \textbf{Chao Peng}, Changqian Yu, Jingbo Wang, Xu Liu, Gang Yu, Wei Jiang\newline
\emph{CVPR, 2019}

\href{https://arxiv.org/pdf/1808.00897.pdf}{BiSeNet: Bilateral Segmentation Network for Real-time Semantic Segmentation}\newline
Changqian Yu, Jingbo Wang, \textbf{Chao Peng}, Changxin Gao, Gang Yu, Nong Sang\newline
\emph{ECCV, 2018}

\href{https://arxiv.org/pdf/1804.03821.pdf}{ExFuse: Enhancing Feature Fusion for Semantic Segmentation}\newline
Zhenli Zhang, Xiangyu Zhang, \textbf{Chao Peng}, Dazhi Cheng, Jian Sun\newline
\emph{ECCV, 2018}

\href{https://arxiv.org/pdf/1804.06215.pdf}{DetNet: A Backbone network for Object Detection} \newline
Zeming Li, \textbf{Chao Peng}, Gang Yu, Xiangyu Zhang, Yangdong Deng, Jian Sun  \newline
\emph{ECCV, 2018}

\href{https://arxiv.org/pdf/1711.07240.pdf}{MegDet: A Large Mini-Batch Object Detector\qquad } \newline
\textbf{Chao Peng}, Tete Xiao, Zeming Li, Yuning Jiang, Xiangyu Zhang, Kai Jia, Gang Yu, \newline
Jian Sun \newline
\emph{CVPR spotlight, 2018}

\href{https://arxiv.org/pdf/1804.09337.pdf}{Learning a Discriminative Feature Network for Semantic Segmentation}\newline
Changqian Yu, Jingbo Wang, \textbf{Chao Peng}, Changxin Gao, Gang Yu, Nong Sang
\newline
\emph{CVPR, 2018}

\href{https://arxiv.org/pdf/1703.02719.pdf}{Large Kernel Matters -- Improve Semantic Segmentation by Global Convolutional Network}
\textbf{Chao Peng}, Xiangyu Zhang, Gang Yu, Guiming Luo, Jian Sun \newline
\emph{CVPR, 2017}

\section{Arxiv论文}
\href{https://arxiv.org/pdf/1711.07264.pdf}{Light-Head R-CNN: In Defense of Two-Stage Object Detector} \newline
Zeming Li, \textbf{Chao Peng}, Gang Yu, Xiangyu Zhang, Yangdong Deng, Jian Sun  \newline
\emph{Arxiv, 2017}


%%% This piece of code has been commented by Karol Kozioł due to biblatex errors. 
% 
%\printbibsection{article}{article in peer-reviewed journal}
%\begin{refsection}
%  \nocite{*}
%  \printbibliography[sorting=chronological, type=inproceedings, title={international peer-reviewed conferences/proceedings}, notkeyword={france}, heading=subbibliography]
%\end{refsection}
%\begin{refsection}
%  \nocite{*}
%  \printbibliography[sorting=chronological, type=inproceedings, title={local peer-reviewed conferences/proceedings}, keyword={france}, heading=subbibliography]
%\end{refsection}
%\printbibsection{misc}{other publications}
%\printbibsection{report}{research reports}

\end{document}
